%%%%%%%%%%%%%%%%%%%%%%%%%%%%%%%%%%%%%%%%%
% a0poster Landscape Poster
% LaTeX Template
% Version 1.0 (22/06/13)
%
% The a0poster class was created by:
% Gerlinde Kettl and Matthias Weiser (tex@kettl.de)
% 
% This template has been downloaded from:
% http://www.LaTeXTemplates.com
%
% License:
% CC BY-NC-SA 3.0 (http://creativecommons.org/licenses/by-nc-sa/3.0/)
%
%%%%%%%%%%%%%%%%%%%%%%%%%%%%%%%%%%%%%%%%%

%----------------------------------------------------------------------------------------
%	PACKAGES AND OTHER DOCUMENT CONFIGURATIONS
%----------------------------------------------------------------------------------------

\documentclass[a0, landscape]{a0poster}
%\usepackage[a0, papersize={36in,36in},  total={30in, 30in}]{geometry}

\usepackage{multicol} % This is so we can have multiple columns of text side-by-side
\columnsep=100pt % This is the amount of white space between the columns in the poster
\columnseprule=3pt % This is the thickness of the black line between the columns in the poster

\usepackage[svgnames]{xcolor} % Specify colors by their 'svgnames', for a full list of all colors available see here: http://www.latextemplates.com/svgnames-colors

\usepackage{times} % Use the times font
%\usepackage{palatino} % Uncomment to use the Palatino font

\usepackage{graphicx} % Required for including images
\graphicspath{{figures/}} % Location of the graphics files
\usepackage{booktabs} % Top and bottom rules for table
\usepackage[font=small,labelfont=bf]{caption} % Required for specifying captions to tables and figures
\usepackage{amsfonts, amsmath, amsthm, amssymb} % For math fonts, symbols and environments
\usepackage{wrapfig} % Allows wrapping text around tables and figures
\usepackage{hyperref} % Allows hyperlinks
\begin{document}

%----------------------------------------------------------------------------------------
%	POSTER HEADER 
%----------------------------------------------------------------------------------------

% The header is divided into three boxes:
% The first is 55% wide and houses the title, subtitle, names and university/organization
% The second is 25% wide and houses contact information
% The third is 19% wide and houses a logo for your university/organization or a photo of you
% The widths of these boxes can be easily edited to accommodate your content as you see fit

\begin{minipage}[b]{1\linewidth}
\veryHuge \color{NavyBlue} \textbf{PPI Prediction via Ensemble Clustering Methods} \color{Black}\\ % Title
% \Huge\textit{An Exploration of Complexity}\\[1cm] % Subtitle
\huge \textbf{Ali Hamza, Haris K. Ladhani, M. Usaid Rahman, Maham S. Patel \& Dr. Humaira Jamshed}\\ % Author(s)
\huge Habib University, DSSE Computer Science\\ % University/organization
\Large \href{mailto:prions.kavish@gmail.com}{prions.kavish@gmail.com}\\ % Website

\end{minipage}
%
% \begin{minipage}[b]{0.55\linewidth}
% \color{DarkSlateGray}\Large \textbf{Contact Information:}\\
% DSSE Computer Science\\ % Address
% Habib University\\
% Gulistan e Johar, Block 18, Pehelwan Goth\\\\
% Phone: +92 317 2453701\\ % Phone number
% Email: \texttt{ah05084@st.habib.edu.pk}\\ % Email address
% \end{minipage}
%
\vspace{1cm} % A bit of extra whitespace between the header and poster content

%----------------------------------------------------------------------------------------

\begin{multicols}{4} % This is how many columns your poster will be broken into, a poster with many figures may benefit from less columns whereas a text-heavy poster benefits from more

%----------------------------------------------------------------------------------------
%	ABSTRAC
%----------------------------------------------------------------------------------------

\color{Navy} % Navy color for the abstract

\begin{abstract}

Protein-Protein Interaction (PPI) is an upcoming field with limitless potential which can help us understanding viral receptor binding and aid drug development among other things. Both in-vitro and in-vivo methods have limitations which is why in-silico methods are gaining popularity among proteomics researchers. However, the current method often falls short offering a room for innovation. Our aim is to create a PPI prediction algorithm that accounts for the topological and biological information whilst making its prediction. Our PPI prediction algorithm will create a more comprehensive and accurate database. We will be using generative models to account for accurate limited data that is currently available, and static and dynamic PPI networks to account for a more realistic PPI representation. We will, thereby, create a prediction algorithm using ensemble clustering methods to better predict PPI using topological and biological information present in the PPI networks.
\end{abstract}

%----------------------------------------------------------------------------------------
%	INTRODUCTION
%----------------------------------------------------------------------------------------

\color{SaddleBrown} % SaddleBrown color for the introduction

\section*{Introduction}


%----------------------------------------------------------------------------------------
%	OBJECTIVES
%----------------------------------------------------------------------------------------

\color{DarkSlateGray} % DarkSlateGray color for the rest of the content

\section*{Main Objectives}

\begin{enumerate}
\item To use generative models to account for limited available valid data.
\item To establish a database that can be used for storing primary and secondary interactions.
\item To make use of dynamic PPI networks to account for real world protien complex behavior.
\item To create a prediction algorithm by using the ensemble clustering framework that also takes into account topographical and biological data when predictin interactions.
\end{enumerate}

%----------------------------------------------------------------------------------------
%	MATERIALS AND METHODS
%----------------------------------------------------------------------------------------

\section*{Materials and Methods}

%------------------------------------------------

\subsection*{Mathematical Section}

%----------------------------------------------------------------------------------------
%	RESULTS 
%----------------------------------------------------------------------------------------

\section*{Results}
%----------------------------------------------------------------------------------------
%	CONCLUSIONS
%----------------------------------------------------------------------------------------

\color{SaddleBrown} % SaddleBrown color for the conclusions to make them stand out

\section*{Conclusions}

\color{DarkSlateGray} % Set the color back to DarkSlateGray for the rest of the content

%----------------------------------------------------------------------------------------
%	FORTHCOMING RESEARCH
%----------------------------------------------------------------------------------------

\section*{Forthcoming Research}

 %----------------------------------------------------------------------------------------
%	REFERENCES
%----------------------------------------------------------------------------------------

\nocite{*} % Print all references regardless of whether they were cited in the poster or not
\bibliographystyle{plain} % Plain referencing style
\bibliography{sample} % Use the example bibliography file sample.bib

%----------------------------------------------------------------------------------------
%	ACKNOWLEDGEMENTS
%----------------------------------------------------------------------------------------

\section*{Acknowledgements}
%----------------------------------------------------------------------------------------

\end{multicols}
\end{document}